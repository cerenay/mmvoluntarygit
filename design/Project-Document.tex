\documentclass[a4paper,10pt]{article}
\usepackage{a4wide,mathptmx,url,lastpage,fancyhdr}






% Don't touch this block:
\pagestyle{fancy}
\rhead{\PROJECT}
\lhead{The Choice Lab---Project document}
\rfoot{\thepage/\pageref{LastPage}}
\cfoot{}
\lfoot{Initials of signatory:}
\renewcommand{\headrulewidth}{0.4pt}
\renewcommand{\footrulewidth}{0.4pt}
\begin{document}

% UPDATE THE PROJECT NAME HERE. NO PROJECT CAN BE ESTABLISHED WITHOUT A PROJECT NAME.
\newcommand{\PROJECT}{\textbf{mmvoluntary}}
\sloppy


% Don't touch this part.
\begin{small} \noindent This project document establishes mmvoluntary as a
project in The Choice Lab,  it authorizes establishment of a version control
repository, and forms the basis for The Choice Lab administrative procedures and
audits. It can be used for establishing an NHH account (\emph{analysenr}) if accompanied by a
budget. 
\end{small}


\section*{Central project data}

% Leave the "\PROJECT\" unchanged, but add the titles, Fill in
% accounting codes and grant numbers as they become available.
% Project code (e.g mm1), responsible person, contact person and
% participants should always be filled in.

\begin{tabular}{rp{10cm}}
Project code and name:   &   \PROJECT\  -- Performance Under In/consistent intrinsic and external motives \\
NHH accounting code:     & XXXX \\
Grant reference numbers: & FILL IN NFR-CODE OR OTHER GRANT DETAILS \\
Responsible for data: & Erik {\O}. S{\o}rensen \\
Contact person:       & Fehime Ceren Ay and Xiaogeng Xu \\
Department:            & Department of Economics              \\
Project participants:    & Fehime Ceren Ay and Xiaogeng Xu \\
Project period:          & 24.04.2017 -- 31.12.2017 \\
\end{tabular}


%Don' touch this part
\medskip
\begin{small}
  \noindent Note that a Ph.D. student cannot be responsible for data
  in the \emph{personopplysningslov} sense. The contact person can be
  relied on to know the day-to-day details of the project.
\end{small}



% Under project description, add a one/two/three paragraph description of project, with a
% focus on the data collection, but also mention the (main) research question. 

\section*{Project description}

Incentives are commonly agreed to have effects on performance. Recent research has shown that performance varies among different incentive schemes when people self select into their preferred incentive schemes although the innate ability of certain tasks in those literature does not vary much among people. Since the incentives are endogenous in these existing literature, it is not obvious whether the observed difference is driven by imposed incentive schemes, or intrinsic preferences for incentive schemes. In other words, the impacts of one's voice in choosing incentives on performance are still under knowledge so far. The so-called voice means the extent to which one can control the incentive assigned to him/her. One may have full control to choose the incentive s/he prefers, or have no voice so as to follow an externally imposed incentive, or have partial control such that the incentive for him/her would be determined by voting. Would the magnitude of self control on incentives influence performance? This is an interesting question for parenting, company management, as well as political institutions. Therefore, this idea attempts to explore how the magnitude of self control in choosing incentives would influence one's performance given one's preference for incentives.

\bigskip  

There are three research questions. First, whether people perform differently when they have full or partial control in choosing incentives given they are assigned same incentives and have some preferences for incentives. Second, whether people with different preferences for incentives would perform differently under same external incentive. Third, whether people with same preference for incentives would perform differently under different external incentives.

% Fill inn YES / NO (or the expected data of completion).
\begin{description}
\item[Is data collection finalized (y/n):] NO.
\item[If not, for when is data collection planned:] November 2017
\item[If not, when is data collection estimated to be finished:] 30th November 2017
\item[Is project finalized (y/n):] NO.
\end{description}

\section*{Data considerations}

% The part below is to summarize status of the project in terms of the personal data act 
% and other 
\begin{description}
\item[Personal data (\emph{personopplysninger}) (y/n):] NO.
\item[Sensitive data? (y/n):] NO
\item[Reasoning for categorization of data:] The data of this project will be collected from a standard lab experiment. All participants will finish simple computing task and get payment according to their performance. Besides the math performance, data collected may include risk preferences, age, gender, education, math enjoyment, competitiveness and basic math score. So no sensitive data will be involved in this project.
\item[Notification sent NSD (y/n):] NO 
\item[Data is moved to secure server (y/n):] YES
\item[Data is submitted to NSD (y/n):] NO
\item[Comments:] 
\end{description}

% Don't touch this part:
\medskip
\begin{small}
  \noindent For an explanation of these terms, see NSD web pages (such
  as \url{http://www.nsd.uib.no/nsd/english/pvo.html}) or contact the
  Choice Lab compliance officer for a discussion of your particular
  case.
\end{small}





% Uncomment this section if/when there are published papers based on the project.
% \section*{Output}
% \begin{enumerate}
% \item LIST WORKING PAPERS OR PUBLICATIONS IN THE LIST HERE
% \end{enumerate}


\section*{History of document}

% The enumeration list below should have one entry for each time substantial changes are made
% to the document -- such as documentation of when notification was sent NSD,
% when data was moved to secure server,  participants are added, data are submitted to the NSD or 
\begin{enumerate}
\item Document established [for already active project (?)] (Apr. XX, 2017).
\end{enumerate}

\section*{Signatures}
\begin{tabular}{ccc}
&            &           \\
&            &           \\
& Fehime Ceren Ay, Xiaogeng Xu         & Erik {\O}. S{\o}rensen       \\
& (Responsible for  \PROJECT\  data)      & (For The Choice Lab,  \\  
&             & compliance officer) \\
Date and place: &             \\   
\end{tabular}



\end{document}